\documentclass[10pt,a4paper,landscape]{report}
\usepackage[utf8]{inputenc}
\usepackage{amsmath}
\usepackage{amsfonts}
\usepackage{amssymb}
\usepackage{geometry}

\begin{document}
\chapter{Rules}
This chapter describes some development rules.

\section{Basic C++}
The identifier should be named in these ways:\\
\begin{tabular}{ll}
ClassSomething & First character and every word in uppercase for a \emph{class}\\
identifierSomething & camelCase for all local variables and \emph{members} (notice: no m\_ prefix for texmaker compatibility)\\
headersomething.h & All header in lowercase \\
headersomething.cpp & sources, too \\
HEADERSOMETHING\_H & include guards in uppercase \\
SOME\_CONSTVALUE & const/enum values in uppercase ?????? (with prefix??? without???)
\end{tabular}

Order the include directives from high level to low level, e.g.:\\
\begin{verbatim}
#include "texmaker.h"

#include <qt>
\end{verbatim}
instead of the other way around

Insert only things in texmaker.cpp or smallUsefulFunction.h if it is really necessary,
better create a new .cpp/.h if you can't find one to put it in

\section {Compatibility}
Don't use QT 4.4 or 4.5, only functions of Qt4.3.\\
If they are really necessary you could wrap them in \#if, like: \\
\verb+#if QT_VERSION >= 0x040500+ für Qt 4.5.0\\
\verb+#if QT_VERSION >= 0x040400+ für Qt 4.4.0\\

But this prevents linking with older qts, so it is better  to call hasAtLeastQt(4,5) of smallAndUseful, to check and then call the methods dynamically.\\
Use QMetaObject::invokeMethod for Slots/Signals and QObject::setProperty for properties

Some can be easily replaced, these include:

\begin{center}
\begin{tabular}{lll}
example & use instead & remark\\
\hline \\
QFormLayout & QGridLayout\\
Q(String)List.append(Q(String)List)  &  $<<$ & This doesn't apply to .append(T) \\
Q(String)List.length  &  QList.count() \\
Q(String)List.removeOne   &  \verb!if (i=QList.indexOf()>0) QList.removeAt(i)! \\
QRegExp.cap const  & QRegExp.cap & cap has first been made const in 4.5 \\
\end{tabular}
\end{center}

Those don't exists in old qt, but should be used for the newer version, so they must be included in \#if.
(min version is the minimum version where it can be used)
\begin{tabular}{llll}
example & min version & possible replacement\\
QPalette::ToolTipBase & 4.4 & QPalette::AlternateBase\\
QPalette::ToolTipText & 4.4 & QPalette::Text
\end{tabular}

\chapter{TODO}

\begin{verbatim}
=====texmaker: 
1.9 changes:
texmaker.cpp: many changes
symbollistwidget: initialized hidden (???)
webpublishdialog: it always used it own settings!

bug: 
\begin{envi- einfügen und rückgängig machen plaziert cursor an falscher column position => schreiben zerstört rückgängig
* Koperien von JabRef (STRG+C) und Einfügen in Notepad++ funktioniert. Einfügen in TexMakerX funktioniert nicht.
not active, empty placeholders
\frac denominator placeholder bug (ein zeichen im placeholder)
spellchecking nach hspace, vspace?
placeholder selection with cursor in it
placeholder selection from codesnippet doesn't work if the placeholder is in the last line and text comes behind it (e.g. \begin{environment-name}
\bar{\ geht nicht, markiert \ und verhindert tippen
enter in structure view
complete selection deletion of placeholder
movement bug, mouse moving right in the same line (probably fixed in new qce)
ausschneiden stört cut
completion crash?? spellchecker crashes??
latex-*.cwl als standard, duplikate, \setminus, 
escape geht nicht immer
next/prev doc remove focus
sprung zur warnung statt error (include with master only?), don't jump away from log file
textanalyse count übersetzungsfehler
Die BibLaTeX-Befehle \parencite und \textcite fehlen

@MastersThesis{Marold,
author = {Alexander Marold},
title = {Rekonstruktion 
der globalen 
Ereignisreihenfolge aus lokalen Logdateien},
school = {Heinrich-Heine-Universität Düsseldorf},
year = {2008},
type = {Bachelorarbeit},
}

 Ist es wirklich so, dass Wärter ''so`` und nicht ``so'' umschlossen
     kein zurückscrollen des log view bei neuem log
     web dialog versteht kein ?ame
     abspeichern scrolls struktur to markierung, bug or feature
     kein kommentieren von ein/ausgeklappten blöcken
autoindent of \end{unknown}

qce:
allow comment indenation/unindation???
commenting one line to far (up to cursor line even if no text there, ignores selection)

Noch ein paar Stichpunkte zum Thesaurus:
- trimmen, größe beibehalten nach schließen
- (multi) markieren, pfeiltasten
- versteht es deklinierte formen?
- enthalten, startend mit (wie im Everest dictionary)
- eigene Wörter hinzufügen, wie bei der Rechtschreibprüfung (prinzipiell könnte man sogar die Ignoredatei benutzen)
- online lookup mit wortschatz.uni-leipzig.de (vermutlich eine blöde idee, weil es nur für deutsch funktionieren würde)

---------------------------------
jump to label from ref, to ref (selection window?)
goto panel

suchfenster

bib tex in eigener klasse, bibtex hover, bibtex auto checking, bib tex right click correction
\label, \ref right click correction
references to all included files
BIBINPUTS = C:\My Documents\CDMA\Papers\Bib;C:\My Documents\GPS\Papers\Bib

\newcounter
obscure tex features:
\def, \edef, \xdef, \gdef for new commands (to be put into the completion list)
\newcount (=tex counter)

\hspace, \vspace ohne rechtschreibprüfung

* Mir ist nicht klar ob/wann TexMakerX BibTeX aufruft. 

unittest (gui, qeditor, qdocument,..)

unittest (gui, qeditor, qdocument,..)

geometry support

(math) font style icons

preview zoom

detextify

http://homepage.ruhr-uni-bochum.de/Georg.Verweyen/silbentrennung.html

* Weshalb erscheint die Combobox zum Ändern nicht nach einem Doppelklick
auf "Standardkürzel"? - nicht jeder ließt die Tabellenüberschriften :)

* Weshalb enthält die Textvervollständigung nicht die Parameter von
Befehlen? - Ich habe gerade die Anfangsbuchstaben des Parameters eines
\textcite{}-Befehls eingegeben und bei SHIFT+ALT+Space kommt kein Vorschlag


Mehrere Funktionen hinzufügen, um Environments/Blöcke/Sektions zu markieren/löschen, etc. ändern $$ zu \[ \]  oder \begin{..} zu \begin{??} 
(wird um einiges einfacher zu implementieren sein, wenn die QCE 2.3 syntax engine fertig ist)

bracket auto close

dateiname für aktives dokument+spalten nunmmer (forward/inv search)

>>
qnfadefinition.cpp, line 1102, replace

if ( ::match(par, parens.top()) )

with

if ( parens.count() > 1 || ::match(par, parens.top()) )

And similarly change line 1233
<<<<

Tooltips bei Scrollen (Zeile\n part\chapter\section)

.chapter analyse dblclick should search only in chapter 

.structure click selecting chapter title

* Ist es Absicht, dass die Ausgabe von pdflatex nicht angezeigt wird?

highlighted match
update works in O(foundMatches), but it could theoretically be done in O(log foundMatches);
and finding the matches the first time needs O(total lines), but O(visible lines) is feasible)

Langfristig:

So wie es auf der Homepage steht:
<li>Improved recognizing of tool paths, checking of correct tool settings,  custom build actions: planned (although former is finished for miktex+ghostscript)</li>
<li>Integrated LaTeX checker: planned</li>
<li>Integrated grammar checker: planned</li>

.lacheck
. command tooltips

die angezeigten Symbolsammlungen konfigurierbar machen, da


\cite{alle werte aus der geparsten bibtex}

[cite]
Das empfinde ich auch als sehr störend. Die meisten Nutzer erwarten, dass der markierter Text mit der Maus verschoben wird und nicht kopiert.
Verschieben wird üblicherweise nur aktiviert, wenn man gleichzeitig die Strg Taste dabei drückt. Beim Schreiben wird wahrscheinlich häufiger Text verschoben als kopiert.
Vielleicht lässt sich die Standardeinstellung ja ändern? 
[/cite]
Das ist wohl Geschmackssache, beim Explorer liest man dauern davon, wie nervig es sei, dass er die Dateien verschiebt statt kopiert.
Aber irgendwann soll es sowieso völlig anpassbar sein, dann kann ich auch eine Option für die Standardaktion einbauen.

Dann noch die gesamten Konfigurationssachen (readsettings, savesettings) von texmakerx.cpp in den configmanager zu verschieben und alle Funktionen, um den Text zu verändert entweder in latexeditorview oder in eine eigene Latexparserklasse (wobei ich mir noch nicht sicher bin, ob die sinnvoll ist).
Irgendwann soll texmakerx.cpp unter 1000 Zeilen liegen...








--------qcodeedit:

search something below the current visible area, scroll up, search again, press arrow down, the search result is drawn selected although it isn't

\end{verbatim}

\end{document}

