\documentclass[10pt,a4paper,landscape]{report}
\usepackage[utf8]{inputenc}
\usepackage{amsmath}
\usepackage{amsfonts}
\usepackage{amssymb}
\usepackage{geometry}

\begin{document}
\chapter{Rules}
This chapter describes some development rules.

\section{Basic C++}
The identifier should be named in these ways:\\
\begin{tabular}{ll}
ClassSomething & First character and every word in uppercase for a \emph{class}\\
identifierSomething & camelCase for all local variables and \emph{members} (notice: no m\_ prefix for texmaker compatibility)\\
headersomething.h & All header in lowercase \\
headersomething.cpp & sources, too \\
HEADERSOMETHING\_H & include guards in uppercase \\
SOME\_CONSTVALUE & const/enum values in uppercase ?????? (with prefix??? without???)
\end{tabular}

Order the include directives from high level to low level, e.g.:\\
\begin{verbatim}
#include "texmaker.h"

#include <qt>
\end{verbatim}
instead of the other way around

Insert only things in texmaker.cpp or smallUsefulFunction.h if it is really necessary,
better create a new .cpp/.h if you can't find one to put it in

\section {Compatibility}
Don't use QT 4.4 or 4.5, only functions of Qt4.3.\\
If they are really necessary you could wrap them in \#if, like: \\
\verb+#if QT_VERSION >= 0x040500+ für Qt 4.5.0\\
\verb+#if QT_VERSION >= 0x040400+ für Qt 4.4.0\\

But this prevents linking with older qts, so it is better  to call hasAtLeastQt(4,5) of smallAndUseful, to check and then call the methods dynamically.\\
Use QMetaObject::invokeMethod for Slots/Signals and QObject::setProperty for properties

Some can be easily replaced, these include:

\begin{center}
\begin{tabular}{lll}
example & use instead & remark\\
\hline \\
QFormLayout & QGridLayout\\
Q(String)List.append(Q(String)List)  &  $<<$ & This doesn't apply to .append(T) \\
Q(String)List.length  &  QList.count() \\
Q(String)List.removeOne   &  \verb!if (i=QList.indexOf()>0) QList.removeAt(i)! \\
QRegExp.cap const  & QRegExp.cap & cap has first been made const in 4.5 \\
\end{tabular}
\end{center}

Those don't exists in old qt, but should be used for the newer version, so they must be included in \#if.
(min version is the minimum version where it can be used)
\begin{tabular}{llll}
example & min version & possible replacement\\
QPalette::ToolTipBase & 4.4 & QPalette::AlternateBase\\
QPalette::ToolTipText & 4.4 & QPalette::Text
\end{tabular}

\chapter{TODO}

\subsection{necessary before 2.1 }

\subsubsection{Bugs}

\subsubsection{others}



\subsection{later}

\subsubsection{Bugs}
\begin{itemize}
\item ctrl+alt+shift env completion doesn't work in the \{\}-bracket pair of \verb+{a}<b>+ 
\item preview shouldn't freeze the whole program for 30s if it praemble compilation failed (at least not every time)
\item line number panel has wrong color on windows (white instead of black) 
\item if you use the environment completion in the unit test results it SOMETIMES adds \verb+\begin{+, but places the cursor before the backslash instead after the bracket (and everything entered later is written backwards)
\item If you have \verb+\begin{+ without a closing \} in the text, placing the cursor behind the \{, creates a placeholder for the whole line after the begin
\item  Auto completion does not correctly count brackets (e.g. if you have \verb+{}}+ an write \verb+{+ before it, a unnecessary closing \verb+}+ is inserted)
\item codesnippt mit placeholder einfügen und rückgängig machen plaziert cursor an falscher column position => schreiben zerstört rückgängig (see failing, commented tests in codesnippet\_t)
\item  env highlighting doesn't work if there are points in the env. name (e.g. \verb+\begin{a..s}+)
\item bracket highlighting bug: write \verb+[]+, place cursor on first column (so the brackets are highlighted), write \%, brackets are now within a comment, but one of them is still highlighted
\item auto replace is not atomic for menu replace (e.g. select text, press shift+alt+f, multiple undos necessary to return to selection)
\item Home/Ende keys don't work (siehe mrunix forum)
\item adding new bib tex entries doesn't always update completer list;; also changing citations in latex doc also require bibtex call and aren't detectedd
\item activating environment completion clears cut buffer (select text, press ctrl+alt+space, select an environment => previously selected text is not inserted)
\item completer doesn't contain citation labels (this means if you have \verb+\cite{abcd|+ with | marking cursor position it doesn't suggest abcdef as citation even if \verb+\cite{abcd}+ is in the list)
\item several newcommand-commands with the same name create several entries in the completer list
\item texworks pdf viewer changes it cursor when shift/alt is pressed, but that only works every 2nd time
\item switching between forward/backward search in the pdf previewer, sometimes doesn't find matches or finds them twice (e.g. in this development.pdf). Possibly a poppler bug
\item tmx nearly crashes if the log files is very large (80 MB, it gets this large if logging is enabled)
\item hint by \verb+\ref+ parses html code. E.g. if you have \begin{verbatim}
<html><body>asas<img src=":/images/splash.png"><b>AAA
\label{key}
</b></body></html>
\end{verbatim} in the latex file
	\item Wrong indentation of \begin{verbatim}
		@MastersThesis{Marold,
		author = {Alexander Marold},
		title = {Rekonstruktion 
		der globalen 
		Ereignisreihenfolge aus lokalen Logdateien},
		school = {Heinrich-Heine-Universität Düsseldorf},
		year = {2008},
		type = {Bachelorarbeit},
		}
	\end{verbatim} in bib files
	\item  qce crashes if the searchpanel is not deleted before the qcodeedit object is deleted (because the search as child of the editor is then automatically deleted after the editor and crashes when it calls the editor to clear it match overlays)
	\item list of open documents still contains an entry, after all open documents were closed
\end{itemize} 

\subsubsection{others}
\begin{itemize}
	\item QCodeEdit: \begin{itemize}
		\item Generate custom highlighting from the used custom completion cwl files. (important for syntax check, if you have a custom math environment you currently have to define it twice, once as custom highlighting(to syntax check multi line environments) and in the cwl (to check single line envs))
		\item Deleting placeholders by pressing repeadetly backspace after one
		\item if replace all is restarted from scope begin, show the count of replaced occurrences before the scope restart and afterwards.
		\item show tooltips when scrolling (line, section, chapter)
		\item RTL support like in texmaker 1.9.0/1 (but there it was removed in 1.9.2) (import fy to unicode)
		\item Auto Closing of \verb+$+ (problem: how to detect if there follows a closing \verb+$+)
	\end{itemize}
	\item Latex Parser: \begin{itemize}
		\item move everything in the \verb+Latexparser+ class
		\item NextWord with cookie state (to recognize multi line verbatim)?? Or read from qce?
		\item Understand some TeX Things (e.g. \verb+^^5c^^5c+ equals \verb+\\+ if not in verbatim)
	\end{itemize}
	\item better structure view: \begin{itemize}
		\item select section title when double clicked of one?		
		\item parsing not loaded files
		\item show section/figure/table numbers
		\item should cache old parsing, don't reparse unmodified files
		\item could use .aux files
		\item should not execute several regex after each other on the same line (custom parser? mixed regex?)	
	\end{itemize}
	\item Synonym dictionary: \begin{itemize}
		\item remember window size
		\item faster contains checks using Tries
		\item support for 
		\item adding own words like in the spell checking 
		\item online lookup (e.g german: wortschatz.uni-leipzig.de)
	\end{itemize}
	\item Labels/References: \begin{itemize}
		\item understand \verb+\nameref, \vref+
		\item allow to define custom commands
		\item 	show all refs using a label and allow easy jumping to them (e.g selection window?)
		\item using a SoundEx suggestion list for wrong references/labels like in the spell checker
		\item a reference/label check dialog like the old spell checking dialog
	\end{itemize}
	\item Citations: \begin{itemize}
		\item unit tests for bibtex parser
		\item show SoundEx suggestion for wrong citatinos
		\item check environment paths (e.g. \verb+BIBINPUTS = C:\My Documents\CDMA\Papers\Bib;C:\My Documents\GPS\Papers\Bib+)
	\end{itemize}
	\item Better new document wizard (look at kile)
	\item Completion \begin{itemize}
	\item  Menu command to complete current/last environment/bracket
	\item \verb+\begin{Umgebungsname}+ should not add \verb+\end{Umgebungsname}+ if that \textbackslash end is already in the next/a later line
	 	\item Open completion list of xyz if cursor is after \verb+\xyz{+ (= ignore \verb+{}+ before cursor)
		\item treat \verb+\+abc\{\%<xyz\%>\} and \verb+\+abc\{\%<def\%>\} as identical and ignore one of them (in the list of possible completions)
		\item Understand \verb+\usepackage}, newcounter, tex: \newcount \newfont \def, \edef, \xdef, \gdef+
		\item Detect completion within tikz/pstricks/... environments and show a list of corresponding tikz/... commands
		\item text completion should show parameters of commands (e.g. reference names) even if they aren't currently used
		\item merging text completion and spell checker cache?
	\end{itemize}
	\item GUI: \begin{itemize}
		\item search for latex commands in the completion package list; list for every available package the contained commands
		\item new tm 1.9 output view (icons left instead of tabs + next/previous error buttons)
		\item using icons for math font styles (like in text font style)
		\item Customizable symbol lists
		\item combo boxes (dropdown tool buttons) instead of several icons for each command (like 1.9 or kile)
		\item toolbar icon size should be customizable
	\end{itemize}
	\item Preview: \begin{itemize}
		\item allow real zoom (regenerating png)
		\item better block selection (e.g. whole beamer frame)
		\item show errors
		
	\item Internal PDF-Previewer:
	\item Options to jump from source to pdf with synctex (context menu?)
	\item border around magnifier
	\item It should be possible to set a filename in the call command and a destination synctex-line
	\item Shortcuts of pdf previewer should be customizable (managedMenu)
	\item Toolbars should also be customizable
	\item There should be an continuous display mode (scrollbar.size = pageheight*pagecount, and currentpage = scrollbar.pos/pageheight).
	\item options menu entry of the pdf previewer should open the config dialog on a preview page
	\end{itemize}
	\item  Checker: \begin{itemize}
		\item Way to disable repetition checker (only in math. envs., globally, for a certain word)
		\item Text analyse should have a list of environments to ignore and if it should read math/envs
		\item Silbentrennung: \verb+http://homepage.ruhr-uni-bochum.de/Georg.Verweyen/silbentrennung.html+
		\item Respect \verb+\verb+ or even multiline \verb+\begin{verbatim}+
		\item Grammar Checker
		\item LaTeX Checker	(calling lacheck?)
	\end{itemize}
	\item Log files:\begin{itemize}
	\item if there are several errors, should it jump to the first or the one after the current cursor position? If the errors are in different files, should it jump to another file? which one? Should it jump to warnings? 	(current in r500 it only jumps to the next error in the same file) 
		\item More log: What do we do with errors in other files when we are not in master mode? 
		\item Is it a good idea to change the tab from ">log file"< to ">messages"< if another tex file is opened and we are not in master mode?
		\item don't jump away from log file (what did that mean???????)
	\end{itemize}
	\item Building \begin{itemize}
		\item Is the output of (pdf)latex shown? should it? Show if it crashes (waiting for pipe, monitor cpu usage)
		\item Improved recognizing of tool paths, checking of correct tool settings,  custom build actions: planned (although former is finished for miktex+ghostscript)
		\item better browser detection in WebPublish dialog 
		\item running arbitrary commands as CMD\_UNKNOWN
		\item better user command (allow to run the other programs as intern://latex, intern://view/dvi, intern://showlog ?)
		\item if the parameters of the user and the program differ, which should be used? (that problem is the reason, ghostscript have to be the plain program name)
		\item . command tooltips (right section?)
	\end{itemize}
	\item  Scripting
	\begin{itemize}
	\item qdocumentsearch access
	\item universalinpuit dialog
	\end{itemize}
	\item Plugins
	\item Recognizing hand written symbols (online lookup in detextify or Windows 7 mathpanel)
	\item New Tabular Dialog (of kile)
	\item Having functions to select environments/blocks/sections, changing \verb+$$+ to \verb+\[\]+, or to \verb+\begin{something}+ (waiting for qce2.3 syntax engine)
	\item single instance: open the documents in the order the second instances were called instead in the order theirs messages arrive in the first instance (using a queue like described in 2877037)
	\item Unit Tests (gui, qeditor, qdocument,..)
\end{itemize}


\end{document}

